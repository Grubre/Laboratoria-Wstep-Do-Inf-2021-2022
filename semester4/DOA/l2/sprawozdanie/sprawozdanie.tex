\documentclass{report}
\usepackage{polski}
\usepackage[utf8]{inputenc}
\usepackage{float}
\usepackage{graphicx}
\usepackage{caption}
\usepackage{subcaption}
\usepackage{ragged2e}
\usepackage{amsmath}
\usepackage{blindtext}
\usepackage{hyperref}
\begin{document}
\author{Jakub Ogrodowczyk}

\section*{Zadanie 1}
Niech:\\
$f$ - tablica, gdzie
$f_i$ = maksymalna ilość paliwa jaką może dostarczyć firma $j$\\
$g$ - tablica, gdzie
$g_i$ = wymagana ilość paliwa na lotnisku $i$\\
$c$ - macierz, gdzie
$c_{ij}$ = koszta dostarczenia galonu paliwa na lotnisko $i$ przez firmę $j$
\subsection*{zmienne decyzyjne}
\(x\) - macierz, gdzie
\(x_{ij}\) = ilość galonów paliwa, które na lotnisko \(i\) dostarczy firma \(j\)
\subsection*{ograniczenia}
$\forall i \forall j : x_{ij} \ge 0$\\
$\forall j : \sum_{i} x_{ij} \le f_j$\\
$\forall i : \sum_{j} x_{ij} = g_i$
\subsection*{funkcja celu}
$\min \sum_{i} \sum_{j} x_{ij} c_{ij}$
\subsection*{wyniki}
Dla danych:\\
$f = (27500, 550000, 660000)$\\
$g = (11000, 220000, 330000, 440000)$\\
$c = \begin{pmatrix}
    10 & 7 & 8 \\
    10 & 11 & 14 \\
    9 & 12 & 4 \\
    11 & 13 & 9
    \end{pmatrix}$\\\\
mamy:\\
$x = \begin{pmatrix}
0 & 110000 & 0 \\
165000 & 55000 & 0 \\
0 & 0 & 330000 \\
110000 & 0 & 330000
\end{pmatrix}$\\
cost $= 8525000$

\section*{Zadanie 2}
Niech:
G = (N,A), gdzie\\
N - zbiór wierzchołków (miast),\\
A - zbiór krawędzi (połączeń),\\
c - funkcja kosztu,\\
t - funkcja czasu,\\
\(T_{max}\) - górne ograniczenie czasu \\
$i^\circ$ - start\\
$j^\circ$ - cel\\
Graf i funkcje c oraz t modeluje za pomocą macierzy\\
\subsection*{zmienne decyzyjne}
\(x\) - macierz, gdzie
$x_{ij} = \begin{cases} 1, & \text{jeśli poszliśmy drogą od i do j} \\ 0, & \text{w p. p.} \end{cases}$\\
\subsection*{ograniczenia}
$\forall i : \sum_{j\in S(i)}x_{ij} - \sum_{j\in P(i)}x_{ji} = \begin{cases} 1 : & i = i^\circ \\ -1 : & i = j^\circ \\ 0 : & \text{w. p. p.} \end{cases}$\\
$\sum_{i}\sum_{j}x_{ij}t_{ij}\leq T_{max}$
\subsection*{funkcja celu}
$\min\sum_{i}\sum_{j}x_{ij}c_{ij}$
\subsection*{wyniki}
Dla danych:\\
$|N| = 10$\\
Graf:\\
\includegraphics*[scale=0.58]{../graph.png}\\
$i^\circ = 1$\\
$j^\circ = 8$\\
$T_{max} = 20$\\
mamy:\\
ścieżka: $1 \rightarrow 10 \rightarrow 6 \rightarrow 7 \rightarrow 8$\\
czas: 16\\
koszt: 99\\
Po usunięciu ograniczeń na całkowitoliczbowość mamy:\\
$x = \begin{pmatrix}
 0 & 0 &  0 &  0 &  0 &  0 &       0 &       0 &       0.363636 & 0.636364\\
 0 & 0 &  0 &  0 &  0 &  0 &       0 &       0 &       0 &       0\\
 0 & 0 &  0 &  0 &  0 &  0 &       0 &       0 &       0 &       0\\
 0 & 0 &  0 &  0 &  0 &  0 &       0 &       0 &       0 &       0\\
 0 & 0 &  0 &  0 &  0 &  0 &       0 &       0 &       0 &       0\\
 0 &  0 &  0 &  0 &  0 &  0 &       0.636364  0 &       0 &       0\\
 0 &  0 &  0 &  0 &  0 &  0 &       0 &       0.636364  0 &       0\\
 0 &  0 &  0 &  0 &  0 &  0 &       0 &       0 &       0 &       0\\
 0 &  0 &  0 &  0 &  0 &  0 &       0 &       0.363636  0 &       0\\
 0 &  0 &  0 &  0 &  0 &  0.636364  0 &       0 &       0 &       0
\end{pmatrix}$\\
Gdy do tego usuniemy ograniczenia na czas, nasze rozwiązanie
ponownie jest akceptowalne (ponadto optymalniejsze od pierwotnego):\\
ścieżka: $1 \rightarrow 9 \rightarrow 8$\\
koszt: 71.0
czas: 27


\section*{Zadanie 3}
Niech:\\
$r_{\text{min}}$ - macierz, gdzie
$r_{\text{min}_{ij}}$ = minimalna ilość radiowozów przydzielonych do dzielnicy i dla zmiany j\\
$r_{\text{max}}$ - macierz, gdzie
$r_{\text{max}_{ij}}$ = maksymalna ilość radiowozów przydzielonych do dzielnicy i dla zmiany j\\
$z_{\text{min}}$ - tablica, gdzie
$z_{\text{min}_{i}}$ = minimalna ilość radiowozów podczas zmiany $i$\\
$p_{\text{min}}$ - tablica, gdzie
$p_{\text{min}_{i}}$ = minimalna ilość radiowozów w dzielnicy $i$\\
\subsection*{zmienne decyzyjne}
\(x\) - macierz, gdzie
\(x_{ij}\) = ilość radiowozów w dzielnicy $i$ podczas zmiany $j$
\subsection*{ograniczenia}
$\forall i \forall j : r_{min_{ij}} \le x_{ij} \le r_{max_{ij}}$\\
$\forall j : \sum_{i} x_{ij} \ge z_{min_j}$\\
$\forall i : \sum_{j} x_{ij} \ge p_{min_i}$\\
\subsection*{funkcja celu}
$\min \sum_{i} \sum_{j} x_{ij}$
\subsection*{wyniki}
Dla danych:\\
$r_{\text{min}} = \begin{pmatrix}
    2 & 4 & 3 \\
    3 & 6 & 5 \\
    5 & 7 & 6 
    \end{pmatrix}$\\
$r_{\text{max}} = \begin{pmatrix}
    3 & 7 & 5 \\
    5 & 7 & 10 \\
    8 & 12 & 10
    \end{pmatrix}$\\
$z_{\text{min}} = (10, 20, 18)$\\
$p_{\text{min}} = (10, 14, 13)$\\
mamy:\\
$x = \begin{pmatrix}
2 & 7 & 5 \\
3 & 6 & 7 \\
5 & 7 & 6 
\end{pmatrix}$\\
ilość radiowozów $ = 48$

\section*{Zadanie 4}
Niech:\\
$n\times m$ - wymiary macierzy $T$ (ilość wierszy, ilość kolumn)\\
$T$ - macierz, gdzie
$T_{ij} = \begin{cases} 1, & \text{jeśli jest w tym polu kontener} \\ 0, & \text{w p. p.} \end{cases}$\\
$k$ - odległość na jaką widzą kamery
\subsection*{zmienne decyzyjne}
$x$ - macierz, gdzie
$x_{ij} = \begin{cases} 1, & \text{jeśli umieszczamy w tym polu kamerę} \\ 0, & \text{w p. p.} \end{cases}$\\
\subsection*{ograniczenia}
$\forall i \forall j : (T_{ij} = 1) \implies (\sum_{l = i-k}^{i+k}x_{lj} + \sum_{l = j-k}^{j + k}x_{il})\ge1$\\
$\forall i \forall j : T_{ij} + x_{ij} \le 1$
\subsection*{funkcja celu}
$\min \sum_{i}\sum_{j}x_{ij}$
\subsection*{wyniki}
\textbf{Dla danych:}\\
$n = 10$\\
$m = 10$\\
ilość kontenerów = $21$\\
$T = \begin{pmatrix}
    0 & 0 & 0 & 0 & 0 & 0 & 0 & 0 & 0 & 0 \\
    0 & 0 & 0 & 0 & 1 & 0 & 0 & 0 & 0 & 0 \\
    0 & 0 & 0 & 0 & 1 & 1 & 1 & 0 & 0 & 0 \\
    1 & 1 & 0 & 0 & 1 & 0 & 0 & 0 & 0 & 0 \\
    0 & 0 & 0 & 0 & 1 & 1 & 0 & 0 & 0 & 0 \\
    0 & 0 & 0 & 0 & 0 & 0 & 0 & 0 & 0 & 0 \\
    0 & 0 & 0 & 0 & 0 & 1 & 1 & 1 & 0 & 0 \\
    0 & 0 & 0 & 0 & 0 & 0 & 0 & 1 & 1 & 1 \\
    1 & 1 & 1 & 1 & 1 & 0 & 0 & 0 & 0 & 0 \\
    0 & 0 & 1 & 0 & 0 & 0 & 0 & 0 & 0 & 0
    \end{pmatrix}$\\    
\textbf{k = 2:}\\
ilość kamer = $10$\\
$x = \begin{pmatrix}
    0 & 0 & 0 & 0 & 0 & 0 & 0 & 0 & 0 & 0 \\
    0 & 0 & 0 & 0 & 0 & 1 & 0 & 0 & 0 & 0 \\
    0 & 0 & 0 & 1 & 0 & 0 & 0 & 0 & 0 & 0 \\
    0 & 0 & 1 & 0 & 0 & 0 & 0 & 0 & 0 & 0 \\
    0 & 0 & 0 & 0 & 0 & 0 & 1 & 0 & 0 & 0 \\
    0 & 0 & 0 & 0 & 0 & 0 & 0 & 0 & 0 & 0 \\
    0 & 1 & 1 & 0 & 0 & 0 & 0 & 0 & 0 & 1 \\
    0 & 0 & 0 & 0 & 0 & 0 & 1 & 0 & 0 & 0 \\
    0 & 0 & 0 & 0 & 0 & 1 & 0 & 0 & 0 & 0 \\
    1 & 0 & 0 & 0 & 0 & 0 & 0 & 0 & 0 & 0
    \end{pmatrix}$\\
\textbf{k = 5:}\\
ilość kamer = $5$\\
$x = \begin{pmatrix}
    0 & 0 & 0 & 0 & 0 & 0 & 0 & 0 & 0 & 0 \\
    0 & 0 & 0 & 0 & 0 & 0 & 0 & 0 & 0 & 0 \\
    0 & 0 & 0 & 0 & 0 & 0 & 0 & 0 & 0 & 0 \\
    0 & 0 & 0 & 0 & 0 & 1 & 0 & 0 & 0 & 0 \\
    0 & 0 & 0 & 0 & 0 & 0 & 0 & 0 & 0 & 0 \\
    0 & 0 & 0 & 0 & 0 & 0 & 0 & 0 & 0 & 0 \\
    0 & 0 & 0 & 0 & 1 & 0 & 0 & 0 & 0 & 0 \\
    0 & 0 & 0 & 0 & 0 & 0 & 1 & 0 & 0 & 0 \\
    0 & 0 & 0 & 0 & 0 & 1 & 0 & 0 & 0 & 0 \\
    0 & 1 & 0 & 0 & 0 & 0 & 0 & 0 & 0 & 0
    \end{pmatrix}$\\    
\section*{Zadanie 5}
Niech:\\
$s$ - tablica, gdzie
$s_i$ = cena sprzedażu wyrobu $i$\\
$c_t$ - tablica, gdzie
$c_{t_i}$ = koszt pracy maszyny $i$\\
$c_m$ - tablica, gdzie
$c_{m_i}$ = koszt materiału wyrobu $i$\\
$p$ - tablica, gdzie
$p_i$ = maksymalny tygodniowy popyt na wyrób $i$\\
$t_{\text{max}}$ - ilość godzin przez które maszyny są dostępne w tygodniu\\
$t$ - macierz, gdzie
$t_{ij}$ = czas potrzebny na zrobienie wyrobu $i$ przez maszynę $j$
\subsection*{zmienne decyzyjne}
$x$ - tablica, gdzie
$x_i$ = wyprodukowana ilość wyrobu $i$
\subsection*{ograniczenia}
$\forall i : 0 \le x_i \le p_i$\\
$\forall j : \sum_{i} x_i \frac{t_{ij}}{60} \le t_{\text{max}}$\\
\subsection*{funkcja celu}
$\max \sum_{i} x_i (s_i - c_{m_i}) - \sum_{j} \frac{c_{t_j}}{60} \sum_{i} x_i t_{ij}$
\subsection*{wyniki}
Dla danych:\\
$s = (9,7,6,5)$\\
$c_t = [2,2,3]$\\
$c_m = [4,1,1,1]$\\
$p = [400, 100, 150, 500]$\\
$t_{\max} = 60$\\
$t = \begin{pmatrix}
    5 & 10 & 6 \\
    3 & 6 & 4 \\
    4 & 5 & 3 \\
    4 & 2 & 1
    \end{pmatrix}$\\
mamy:\\
zarobek = $3632.5$\\
$x = (125, 100, 150, 500)$
\end{document}
