\documentclass{article}
\usepackage{polski}
\usepackage[utf8]{inputenc}
\usepackage{float}
\usepackage{graphicx}
\usepackage{caption}
\usepackage{subcaption}
\usepackage{ragged2e}
\usepackage{amsmath}
\usepackage{amssymb}
\usepackage{amsfonts}
\usepackage{blindtext}
\usepackage{hyperref}
\usepackage{listings}
\usepackage{booktabs}
\usepackage{siunitx}
\usepackage{multicol}
\usepackage{fancyvrb}
\usepackage[linesnumbered,ruled,vlined]{algorithm2e}

\begin{document}
\title{Lista 5}
\author{Jakub Ogrodowczyk}
\date{\today}

\maketitle

\section{Opis problemu}
Zadanie polega na rozwiązaniu układu równań liniowych:
\[
    \mathbf{Ax = b}    
\]
dla danej macierzy \textbf{rzadkiej} \( \mathbf{A} \in \mathbb{R}^{n \times n}\) postaci: 

\[
A = 
\begin{pmatrix}
A_1 & C_1 & 0 & 0 & 0 & \cdots & 0 \\
B_2 & A_2 & C_2 & 0 & 0 & \cdots & 0 \\
0 & B_3 & A_3 & C_3 & 0 & \hdots & 0  \\
\vdots & \ddots & \ddots & \ddots & \ddots & \ddots & \vdots \\
0 & \cdots & 0 & B_{\nu-2} & A_{\nu-2} & C_{\nu-2} & 0 \\
0 & \cdots & 0 & 0 & B_{\nu-1} & A_{\nu-1} & C_{\nu-1} \\
0 & \cdots & 0 & 0 & 0 & B_\nu & A_\nu
\end{pmatrix},
\]

gdzie $\mathbf{A_k}$, $\mathbf{B_k}$, $\mathbf{C_k}$ $\mathbf{0} \in \mathbb{R}^{l \times l}$ dla $k=1\hdots v-1$ oraz $v=\frac{n}{l}$.
$\mathbf{A_k}$ jest macierzą gęstą, $\mathbf{0}$ jest macierzą zerową.
Macierz $\mathbf{B_k}$ jest postaci:
\[
B_k = 
\begin{pmatrix}
0 & \hdots & 0  & b_1^k \\
0 & \hdots & 0  & b_2^k \\
\vdots &  & \vdots  & \vdots \\
0 & \cdots & 0  & b_l^k \\
\end{pmatrix}
\]
A macierz $\mathbf{C_k}$:
\[
C_k = 
\begin{pmatrix}
c_1^k & 0 & 0 & \cdots & 0 \\
0 & c_2^k & 0 & \cdots & 0 \\
\vdots & \vdots & \ddots & \ddots & \vdots \\
0 & \cdots & 0 & c_{\ell-1}^k & 0 \\
0 & \cdots & 0 & 0 & c_{\ell}^k
\end{pmatrix}
\]




\end{document}